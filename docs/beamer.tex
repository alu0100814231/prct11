\documentclass{beamer}
\usepackage[spanish]{babel}
\usepackage[utf8]{inputenc}
\usepackage{graphicx} %para añadir imagen

\title[Presentacion con Beamer]{Número $\pi$} %No lo ponemos dentro del begin porque ya están predefinidos
\author[Técnicas experimentales]{Bianca E. Kennedy Giménez}
\date[25/04/14]{25 de abril de 2014}

\usetheme{Madrid}
\begin{document}

\begin{frame} %creamos una página solo para el título
\titlepage
\end{frame}

\begin{frame}
\frametitle{Índice}
\tableofcontents[pausesections] %con este comando hacemos que ponga en esta transparencia todas las secciones  [opcion para que aparezcan con pausa en casa clic]
\end{frame}

\section{Primera sección} %Abrimos una sección


\begin{frame}
\frametitle{Introducción} %No es posible sin abrir una sección
$\pi$ es la relación entre la longitud de la circunferencia y su diámetro, en geometría euclidiana. Es un número irracional y una de las constantes matemáticas más importantes. El valor númerico de $\pi$, tuncando a sus primeras cifras, es el siguiente:

\begin{displaymath} %Se utiliza para ecuaciones y no hacer falta poner simbolo de dolar
    \pi \approx 3,14159265358979323846. 
\end{displaymath}
\end{frame}



\section{Segunda sección}

\begin{frame}
\frametitle{Historia}
Vamos a hacer un ejemplo:
\begin{itemize}
\item Antiguo egipto\pause
\item Mesopotamia\pause
\item Antiguedad clásica\pause
\item Matemática china\pause
\item Matemática india\pause
\footnote{Hay muchas más}
\end{itemize}
\end{frame}

\subsection{Antiguo egipto}
\begin{frame}
\begin{itemize}
\frametitle{Historia.Antiguo egipto.}
 \item El valor aproximado de $\pi$ en las antiguas culturas se remonta a la época del escriba egipcio Ahmes en el año 1800 a.C., decrito en el papiro Rindh, donde se emplea un valor de $\pi$ afirmando que el área de un cículo es similar a la de un cuadrado cuyo lado es igual al diámetro del círculo disminuido en $\frac{1}{9}$, es decir, igual a $\frac{8}{9}$ del diámetro. En notación moderna: 
 \[S = \pi r^2 \simeq \left ( \frac{8}{9} \cdot d \right )^2 = \frac{64}{81} d^2 = \frac{64}{81} \left(4 r^2\right)\] 
\end{itemize}
\end{frame}

\subsection{Mesopotamia}
\begin{frame}
\begin{itemize}
\frametitle{Historia.Mesopotamia.}
 \item Algunos matemáticos mesopotámicos empleaban, en el cálculo de segmentos, valores de $\pi$ igual a 3, alcanzando en algunos casos valores más aproximados, como el de:
 \[ \pi \approx 3 + \frac{1}{8} = 3,125 \] 
\end{itemize}
\end{frame}

\subsection{Antiguedad clásica}
\begin{frame}
\begin{itemize}
\frametitle{Historia.antiguedad clásica.}
 \item El matemático griego Arquímedes(siglo III, a.C.) fue capaz de determinar el valor de $\pi$ entre el intervalo comprendido por $\frac{3 10}{71}$, como valor mínimo, y $\frac{3 1}{7}$, como valor máximo. Con esta aproximación de Arquímedes se obtiene un valor con un error que oscila entre 0,024$\%$ y 0,040$\%$ sobre el valor real. El método usado por Arquímedes era muy simple y consistía en circunscribir e inscribir polígonos regulares de n-lados en circunferencias y calcular el perímetro de dichos polígonos. Arquímedes empezó con hexágonos circunscritos e inscritos, y fue doblando el número de lados hasta llegar a polígonos de 96 lados.

Alrededor del año 20 d. C., el arquitecto e ingeniero romano Vitruvio calcula $\pi$ como el valor fraccionario $\frac{25}{8}$ midiendo la distancia recorrida en una revolución por una rueda de diámetro conocido.

En el siglo II,Claudio Ptolomeo proporciona un valor fraccionario por aproximaciones:
\[ \pi \simeq \frac{377}{120} = 3{,}1416 \]  
\end{itemize}
\end{frame}

\begin{frame}
\begin{itemize}
\frametitle{Historia.Matemática china.}
 \item El cálculo de $\pi$ fue una atracción para los matemáticos expertos de todas las culturas. Hacia 120, el astrónomo chino Zhang Heng fue uno de los primeros en usar la aproximación $\sqrt {10}$, que dedujo de la razón entre el volumen de un cubo y la respectiva esfera inscrita.
    \[ \sum_{n=0}^{\infty} \frac{(-1)^n}{2n+1} = 1 - \frac{1}{3} + \frac{1}{5} - \dots = \frac{\pi}{4} \] 
 
\end{itemize}
\end{frame}

\begin{frame}
\begin{itemize}
\frametitle{Historia.Matemática india.}
 \item Usando un polígono regular inscrito de 384 lados, a finales del siglo V el matemático indio Aryabhata estimó el valor en 3,1416. A mediados del siglo VII, estimando incorrecta la aproximación de Aryabhata, Brahmagupta calcula $\pi$ como $\sqrt {10}$, cálculo mucho menos preciso que el de su predecesor.

  \[  \pi = \sum_{k=0}^\infty \frac{2(-1)^k\; 3^{\frac{1}{2} - k}}{2k+1} \]
\end{itemize}
\end{frame}


\begin{frame}
\frametitle{Bibliografía}
\begin{thebibliography}
\beamertemplatebookbibitems
\bibitem[Guía Docente,2013]{guia}
Guía docente (año 2013)
{\small $http://gjtsrh.com$}
\end{thebibliography}
\end{frame}

\end{document}             